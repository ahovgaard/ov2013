\section{\texttt{mosmlyac} parser generator code} \label{app:grm}
\begin{lstlisting}[style=MLStyle]
%{
  (* See the Moscow ML manual for the syntax and structure. Roughly:
   *
   * `%``{`
   *     header (* with ML-like comments *)
   *     Any SML code is copied to the beginning of Parser.sml, after the token
   *     datatype declaration.
   * `%``}`
   *     declarations  /* with c-like comments */
   * `%``%`
   *     rules         /* with c-like comments */
   * `%``%`
   *     trailer (* with ML-like comments *)
   *     Any SML code is copied to the end of Parser.sml.
   * EOF
   *
   * The (optional) header and trailer contain ML code to include in the
   * generated file (after the data declaration and at the end).
   *)

  type Pos   = int * int  (* position: (line, column) *)
  type Ident = string     (* identifier *)
  (* Unfortunately, we cannot use these types in the declarations -
   * the code ends up _after_ the data declaration. *)

  (* parse exception *)
  exception ParseError of string * Pos


  (* Get position tuple from Location structure, for error reporting *)
  fun getPos () = case Location.getCurrentLocation () of
                       Location.Loc (l, r) => (l, r)
%}

/* Token type definitions (will often be used in the Lexer) */

/* Tokens use position attribute for demonstration (see below for Lexer)
 * As mentioned, the SML code above ends up _after_ this data declaration,
 * so we cannot use any types defined _above_ at this point of the file.
 */

/* Keywords */
%token <int * int> TProgram   TFunction   TProcedure
%token <int * int> TVar       TBegin      TEnd
%token <int * int> TIf        TThen       TElse
%token <int * int> TWhile     TDo         TReturn

/* Type keywords */
%token <int * int> TArray     TOf
%token <int * int> TInt       TChar       TBool

/* Symbols */
%token <int * int> TSemi      TColon      TComma      TAssign

/* Operations */
%token <int * int> TPlus      TMinus      TTimes      TSlash /* arithmetic */
%token <int * int> TEq        TLess                          /* comparison */
%token <int * int> TAnd       TOr         TNot               /* bool */

/* Parentheses of different kind */
%token <int * int> TLParen    TRParen
%token <int * int> TLCurly    TRCurly
%token <int * int> TLBracket  TRBracket

/* Identifiers */
%token <string * (int * int)> TId 

/* Literals */
%token <int * (int * int)>    TNLit
%token <bool * (int * int)>   TBLit
%token <char * (int * int)>   TCLit
%token <string * (int * int)> TSLit

/* EOF special token */
%token <int * int>            TEOF


/* Precedence and associativity */

/* fixing the dangling-else ambiguity */
%right TThen TElse 

%left TOr
%left TAnd
%left TEq 
%left TLess
%left TPlus TMinus
%left TTimes TSlash


/* high precedence, right associative, unary operator 'not' */
%right TNot

/* fixed the Decs problem with nonassoc */ 
%nonassoc TSemi
%nonassoc TId

/* Start symbol */
%start Prog

/******************************************************************************
 * Types returned by rules below 
 *****************************************************************************/

/* Program structure */
%type <AbSyn.Prog>        Prog
%type <AbSyn.FunDec list> FunDecs
%type <AbSyn.FunDec>      FunDec
%type <AbSyn.StmtBlock>   Block
%type <AbSyn.Dec list>    DBlock
%type <AbSyn.Stmt list>   SBlock
%type <AbSyn.Stmt list>   StmtSeq

/* Variables and parameter declarations, types */
%type <AbSyn.Dec list>    PDecl
%type <AbSyn.Dec list>    Params
%type <AbSyn.Dec>         Dec
%type <AbSyn.Dec list>    Decs
%type <AbSyn.Type>        Type

/* Statements */
%type <AbSyn.Stmt>        Stmt

/* Function and procedure parameters and index lists */
%type <AbSyn.Exp list>    CallParams
%type <AbSyn.Exp list>    Exps

/* L-Values and expressions */
%type <AbSyn.LVAL>        LVal
%type <AbSyn.Exp option>  Ret
%type <AbSyn.Exp>         Exp

%%

/******************************************************************************
 * Rules - a separate start rule is added automatically
 *****************************************************************************/

/* Program structure */
Prog    : TProgram TId TSemi FunDecs TEOF { $4 } /* $4 : AbSyn.FunDec list */

FunDecs : FunDecs FunDec { $1 @ [$2] }
        | FunDec         { [$1] }

FunDec  : TFunction TId TLParen PDecl TRParen TColon Type Block TSemi
            /* Func of Type * Ident * Dec list * StmtBlock * Pos */
            { AbSyn.Func ($7, #1 $2, $4, $8, getPos ()) }
        | TProcedure TId TLParen PDecl TRParen Block TSemi
            /* Proc of Ident * Dec list * StmtBlock * Pos */
            { AbSyn.Proc (#1 $2, $4, $6, getPos ()) }

Block   : DBlock SBlock { AbSyn.Block ($1, $2) }

DBlock  : TVar Decs { $2 } /* $2 : AbSyn.Dec list */
        |           { [] }

SBlock  : TBegin StmtSeq TSemi TEnd { $2 } /* $2 : AbSyn.Stmt list */
        | Stmt                      { [$1] } /* $1 : AbSyn.Stmt */

StmtSeq : StmtSeq TSemi Stmt { $1 @ [$3] }  /* AbSyn.Stmt list */
        | Stmt               { [$1] }


/* Variables and parameter declarations, types */
PDecl  : Params             { $1 } /* AbSyn.Dec list */
       |                    { [] }

Params : Params TSemi Dec   { $1 @ [$3] } /* AbSyn.Dec list */
       | Dec                { [$1] }

Dec    : TId TColon Type    /* AbSyn.Dec of Ident * Type * Pos */
                            { AbSyn.Dec (#1 $1, $3, getPos ()) }
                            
Decs   : Dec TSemi Decs     { $1 :: $3 }
       | Dec TSemi          { [$1] }

Type   : TInt               { AbSyn.Int (getPos ()) }
       | TChar              { AbSyn.Char (getPos ()) }
       | TBool              { AbSyn.Bool (getPos ()) }
       | TArray TOf Type    { AbSyn.Array ($3, getPos ()) }


/* Statements */
Stmt : TId TLParen CallParams TRParen  /* ProcCall of Ident * Exp list * Pos */
                                       { AbSyn.ProcCall (#1 $1, $3, getPos ()) }
     | TIf Exp TThen Block TElse Block
         /* IfThEl of Exp * StmtBlock * StmtBlock * Pos */
         { AbSyn.IfThEl ($2, $4, $6, getPos ()) }
     | TIf Exp TThen Block            
         /* IfThEl of Exp * StmtBlock * StmtBLock * Pos */
         { AbSyn.IfThEl ($2, $4, AbSyn.Block ([], []), getPos ()) }
     | TWhile Exp TDo Block            /* While of Exp * StmtBlock * Pos */
                                       { AbSyn.While ($2, $4, getPos ()) }
     | TReturn Ret                     /* Return of Exp option * Pos */
                                       { AbSyn.Return ($2, getPos ()) }
     | LVal TAssign Exp                /* Assign of LVAL * Exp * Pos */
                                       { AbSyn.Assign ($1, $3, getPos ()) }


/* Function and procedure parameters and index lists */
CallParams : Exps { $1 }
           |      { [] }

Exps       : Exp TComma Exps { $1 :: $3 }
           | Exp             { [$1] }


/* L-Values and expressions */
LVal : TId                          { AbSyn.Var (#1 $1) }
     | TId TLBracket Exps TRBracket { AbSyn.Index (#1 $1, $3) }

Ret  : Exp { SOME $1 } /* AbSyn.Exp option */
     |     { NONE }

Exp  : TNLit { AbSyn.Literal ( AbSyn.BVal (AbSyn.Num (#1 $1)), getPos ()) }
     | TBLit { AbSyn.Literal ( AbSyn.BVal (AbSyn.Log (#1 $1)), getPos ()) }
     | TCLit { AbSyn.Literal ( AbSyn.BVal (AbSyn.Chr (#1 $1)), getPos ()) }
     | TSLit { AbSyn.StrLit (#1 $1, getPos ()) }
     | TLCurly Exps TRCurly { AbSyn.ArrLit ($2, getPos ()) }
     | LVal { AbSyn.LValue ($1, getPos ()) }
     | TNot Exp { AbSyn.Not ($2, getPos ()) }
     | TLParen Exp TRParen { $2 }
     | TId TLParen CallParams TRParen { AbSyn.FunApp (#1 $1, $3, getPos ()) }

     | Exp TPlus  Exp { AbSyn.Plus ($1, $3, getPos ()) }
     | Exp TMinus Exp { AbSyn.Minus ($1, $3, getPos ()) }
     | Exp TTimes Exp { AbSyn.Times ($1, $3, getPos ()) }
     | Exp TSlash Exp { AbSyn.Div ($1, $3, getPos ()) }
     | Exp TEq    Exp { AbSyn.Equal ($1, $3, getPos ()) }
     | Exp TLess  Exp { AbSyn.Less ($1, $3, getPos ()) }
     | Exp TAnd   Exp { AbSyn.And ($1, $3, getPos ()) }
     | Exp TOr    Exp { AbSyn.Or ($1, $3, getPos ()) }

%%

(* mosmlyac trailer *)
\end{lstlisting}
