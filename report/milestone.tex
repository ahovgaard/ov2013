\documentclass[12pt,a4paper]{article}

% use utf8 encoding
\usepackage[utf8]{inputenc}

% math environments (eg. align) and symbols
\usepackage{amsmath, amssymb}

% import graphicx for managing pictures
\usepackage{graphicx}

% import float for positioning using [H] in figures
\usepackage{float}

% import colortbl for coloring tabular cells
\usepackage{colortbl}

% tikz for drawing diagrams
\usepackage{tikz}
\usetikzlibrary{automata,positioning,arrows}

% lstlisting environment for source code
\usepackage{listings}

% listings config for writing context-free grammars
\lstset{
  basicstyle=\ttfamily,
  columns = fixed,
  mathescape,
  xleftmargin=5em,
  literate={->}{$\rightarrow$}{2}
           {alpha}{$\alpha$}{1}
           {delta}{$\delta$}{1}
           {epsilon}{$\varepsilon$}{1}
}

\begin{document}
\title{Introduction to Compilers (OV)\\
       Milestone status report}
\author{Anders Kiel Hovgaard\\
        Daniel Gavin\\
        Rúni Klein Hansen}
\date{December 6, 2013}
\maketitle

\section{Introduction}
The task at hand is to construct a working parser for the language Paladim,
created for the ``Oversætter''-course 2013. The tool used is a parser generator
named mosmlyac.\\
In this milestone report we will focus on getting the parser up and running
while f.ex. evaluation of boolean expressions will be saved for later.


\section{Parser implementation in mosmlyac}


\section{Resolving parse conflicts}
We fixed the dangling-else ambiguity by making the \texttt{then} and
\texttt{else} symbols right-associative and giving them the same precedence,
i.e. by using the following mosmlyac declaration:
\begin{verbatim}
  %right TThen TElse 
\end{verbatim}

\section{Changes to the Paladim grammar}
\begin{lstlisting}
  $Decs$  ->    $Decs$ $Dec$ ;
  $Decs$  ->    $Dec$ ;
\end{lstlisting}
\begin{lstlisting}
  $Decs$    ->   $Dec$ ; $Decs1$
  $Decs1$   ->    $Decs$
  $Decs1$   ->    epsilon
\end{lstlisting}

\section{Testing methodology}
At the current time in the spacetime-continuum we are only able to compile a subset
of the programs included in the \textbf{DATA}-folder. Those programs are:\\
\begin{enumerate}
  \item fibRex.pal
  \item fibWhile.pal
  \item proctest.pal
  \item readtest.pal
  \item shortest.pal
\end{enumerate}

\section{Conclusion}

\begin{thebibliography}{9}
  \bibitem[ICD]{icd}
    Torben Ægidius Mogensen,
    \emph{Introduction to Compiler Design},
    Springer London, first edition, 2011.
\end{thebibliography}

\end{document}
