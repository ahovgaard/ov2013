\documentclass[12pt,a4paper]{article}

% use utf8 encoding
\usepackage[utf8]{inputenc}

% math environments (eg. align) and symbols
\usepackage{amsmath, amssymb}

% import graphicx for managing pictures
\usepackage{graphicx}

% import float for positioning using [H] in figures
\usepackage{float}

% import colortbl for coloring tabular cells
\usepackage{colortbl}

% tikz for drawing diagrams
\usepackage{tikz}
\usetikzlibrary{automata,positioning,arrows}

% lstlisting environment for source code
\usepackage{listings}

% listings config for writing context-free grammars
\lstdefinestyle{GrammarStyle}{
    basicstyle=\ttfamily
  , columns=fixed
  , mathescape
  , keepspaces=true
  , xleftmargin=2em
  , literate={->}{$\rightarrow$}{2}
             {alpha}{$\alpha$}{1}
             {delta}{$\delta$}{1}
             {epsilon}{$\varepsilon$}{1}
}

% listings config for Standard ML code
\lstdefinestyle{MLStyle}{
  %language=ML, % mosmlyac isn't really ML
  basicstyle=\scriptsize,
  numbers=left,
  numberstyle=\tiny,
  frame=tb,
  showstringspaces=false
}

\begin{document}
\title{Introduction to Compilers (OV)\\
       Milestone status report}
\author{Anders Kiel Hovgaard\\
        Daniel Gavin\\
        Rúni Klein Hansen}
\date{December 6, 2013}
\maketitle

\section{Introduction}
The task at hand is to construct a working parser for the language Paladim,
created for the ``Introduction to Compilers''-course (OV) 2013. The tool used is
a parser generator named \texttt{mosmlyac}.\\
In this milestone report we will focus on getting the parser up and running
while e.g. evaluation of boolean expressions will be saved for later.


\section{Parser implementation in \texttt{mosmlyac}}
A large part of this milestone assignment is the implementation of a parser for
the toy language \textsf{Paladim}, using the parser generator \texttt{mosmlyac}.
The code for the \texttt{mosmlyac} parser generator is included in appendix
\ref{app:grm}.\\
\\
The file ``Parser.grm'' declares a number of token types that represent the
different tokens, i.e. entities or objects, of the \textsf{Paladim} language.
The lexical analyser of the compiler, as generated using \texttt{mosmllex} and
specified in ``Lexer.lex'', returns a list of tokens of this type and this is
given as input to the parser. To do this, the lexer code was also changed to
produce the types defined in the new \texttt{mosmlyac} parser, rather than the
ones defined in the LL(1) parser, which basically just involves replacing all
occurences of \emph{LL1Parser} with \emph{Parser}, in the file ``Lexer.lex''.\\
The grammar rules, given in figure 3 of the group project assignment
specification (``GroupProject.pdf''), is rewritten in a \texttt{mosmlyac}
specific way of specifying the grammar of the language, but besides from minor
syntactical differences, they are largely identical.\\
\\
Our main goal was to create the same language as the LL(1) parser, and then 
connect the new parser to the rest the compiler(Changing the Driver.sml to use our parser instead). 
We started with the grammar, which was given with the assignment. We then modified the
grammar to conform with \texttt{mosmlyacc}, and also modified certain areas of
the grammar that was giving conflicts (you can see the conflicts we had in the
'Resolving parse conflicts' section).\\
\\
We also changed some of the given grammar that we thought could be better
expressed with some new grammar, but still did that same thing as the given
grammar (see the 'Changes to the \textsf{Paladim} grammar' section for that).\\
\\
When we were done with the grammar we also had to make sure we used the correct
return types, we used the ``AbSyn.sml'' file to check for the the datatype types
to use, and we also looked in the LL(1) parser to see what types that parser
used for better understanding of the types.


\section{Resolving parse conflicts}
We fixed the dangling-else ambiguity by making the \texttt{then} and
\texttt{else} symbols right-associative and giving them the same precedence,
i.e. by using the following mosmlyac declaration:

\begin{verbatim}
  %right TThen TElse 
\end{verbatim}

\noindent
We had a shift/reduce conflict with the modified productions of the $Decs$
nonterminal (which you can see in the changes to the paladin grammar section).
The way we solved that problem was to force a reduce over a shift by making
\texttt{TId} and \texttt{TSemi} non-associative, which fixed the conflict
problem. \\
\\
We also received countless reduce conflicts from the rewritten operation
expressions (see 'Changes to \textsf{Paladim} grammar' section).  This was
because the parser did not know which reduction it should choose, e.g. 2+3*3,
Plus(2, Times(3,3)) or Times(Plus(2, 3), 3). To fix this we introduced
precedence and associativity for each operation and that fixed the problem.
Please see the appendix for the precedence and associativity assignments for
each operation in the ``Parser.grm'' code.

\section{Changes to the \textsf{Paladim} grammar}
We removed the OP non terminal and instead created all the cases for the
different operations instead.
\begin{lstlisting}[style=GrammarStyle]
  $Exp\;$  ->  $Exp$ $OP$ $Exp$ 
  $OP$   ->  TPlus | TMinus | TTimes | TSlash 
  $OP$   ->  TEq | TLess | TAnd | TOr
\end{lstlisting}
to
\begin{lstlisting}[style=GrammarStyle]
  $Exp$  ->   $Exp$ TPlus  $Exp$
  $Exp$  ->   $Exp$ TMinus $Exp$
  $Exp$  ->   $Exp$ TTimes $Exp$
  $Exp$  ->   $Exp$ TSlash $Exp$
  $Exp$  ->   $Exp$ TEq    $Exp$
  $Exp$  ->   $Exp$ TLess  $Exp$
  $Exp$  ->   $Exp$ TAnd   $Exp$
  $Exp$  ->   $Exp$ TOr    $Exp$ 
\end{lstlisting}
\hfill\\
We decided to remove the left recursion because of a reduce/shift conflict.
\begin{lstlisting}[style=GrammarStyle]
  $Decs$  ->    $Decs$ $Dec$ ; 
  $Decs$  ->    $Dec$ ;
\end{lstlisting}
to
\begin{lstlisting}[style=GrammarStyle]
  $Decs$   ->   $Dec$ ; $Decs$
  $Decs$   ->   $Dec$ ;
\end{lstlisting}


\section{Testing methodology}
At the current point in the space-time continuum we are only able to compile a
subset of the programs included in the \textbf{DATA}-folder. Those programs
are:\\
\begin{enumerate}
  \item fibRex.pal
  \item fibWhile.pal
  \item proctest.pal
  \item readtest.pal
  \item shortest.pal
\end{enumerate}

Also whenever we changed something new in the grammar, e.g. modified some
precedence for the Expression, we would use our grammar script which allows us
to create the AbSyn syntax tree without it being further processed. This was
quite handy since we can actually test things that aren't implemented yet for
the compiler, typecheck, etc. A good example of the usage of the grammar script
was when we had to test the 'or' and 'not' operators. Since the compiler still
does not support these operations, we had to use this script to see if our
precedence was correct - which it is.
\\
\\
Whenever we had made some major changes to the parser, we would run all the
aforementioned pal files in interpreter- and compile-mode to check if everything
still worked. Before we even started coding the grammar for the parser, we saw
how the LL(1) parser handled the aforementioned pal files, and since we are
building a replacement for the LL(1) parser, our new parser should at least
handle the same files and give the same output as the LL parser. It was
therefore fairly easy to see if the parser tests were correct, regarding the pal
files. Pal files like the fibWhile and fibRec were also quite easy to test,
since they took some input and gave some output, which made it alot more easy to
make test cases.


\section{Conclusion}
Using the knowledge of the LL(1) parser (given from the assignment) and that of
yacc, we rewrote the grammar from the given grammar. Using our knowledge of LR
parsers to handle shift/reduce conflicts, we used associativity to handle
reduce/shift conflicts. By testing all of our changes to the parser, we can be
quite certain that it actually works as intended, and even though we didn't do
really structured test cases, we did test quite a lot. We think in the later
parts of the project a more structured testing approach would be better, but we
felt that the tests were quite simple for the parser, and therefore testing of
the already given pal files and our grammarTester script were enough.

\newpage
\appendix
\section{\texttt{mosmlyac} parser generator code} \label{app:grm}
\begin{lstlisting}[style=MLStyle]
%{
  (* See the Moscow ML manual for the syntax and structure. Roughly:
   *
   * `%``{`
   *     header (* with ML-like comments *)
   *     Any SML code is copied to the beginning of Parser.sml, after the token
   *     datatype declaration.
   * `%``}`
   *     declarations  /* with c-like comments */
   * `%``%`
   *     rules         /* with c-like comments */
   * `%``%`
   *     trailer (* with ML-like comments *)
   *     Any SML code is copied to the end of Parser.sml.
   * EOF
   *
   * The (optional) header and trailer contain ML code to include in the
   * generated file (after the data declaration and at the end).
   *)

  type Pos   = int * int  (* position: (line, column) *)
  type Ident = string     (* identifier *)
  (* Unfortunately, we cannot use these types in the declarations -
   * the code ends up _after_ the data declaration. *)

  (* parse exception *)
  exception ParseError of string * Pos


  (* Get position tuple from Location structure, for error reporting *)
  fun getPos () = case Location.getCurrentLocation () of
                       Location.Loc (l, r) => (l, r)
%}

/* Token type definitions (will often be used in the Lexer) */

/* Tokens use position attribute for demonstration (see below for Lexer)
 * As mentioned, the SML code above ends up _after_ this data declaration,
 * so we cannot use any types defined _above_ at this point of the file.
 */

/* Keywords */
%token <int * int> TProgram   TFunction   TProcedure
%token <int * int> TVar       TBegin      TEnd
%token <int * int> TIf        TThen       TElse
%token <int * int> TWhile     TDo         TReturn

/* Type keywords */
%token <int * int> TArray     TOf
%token <int * int> TInt       TChar       TBool

/* Symbols */
%token <int * int> TSemi      TColon      TComma      TAssign

/* Operations */
%token <int * int> TPlus      TMinus      TTimes      TSlash /* arithmetic */
%token <int * int> TEq        TLess                          /* comparison */
%token <int * int> TAnd       TOr         TNot               /* bool */

/* Parentheses of different kind */
%token <int * int> TLParen    TRParen
%token <int * int> TLCurly    TRCurly
%token <int * int> TLBracket  TRBracket

/* Identifiers */
%token <string * (int * int)> TId 

/* Literals */
%token <int * (int * int)>    TNLit
%token <bool * (int * int)>   TBLit
%token <char * (int * int)>   TCLit
%token <string * (int * int)> TSLit

/* EOF special token */
%token <int * int>            TEOF


/* Precedence and associativity */

/* fixing the dangling-else ambiguity */
%right TThen TElse 

%left TOr
%left TAnd
%left TEq 
%left TLess
%left TPlus TMinus
%left TTimes TSlash


/* high precedence, right associative, unary operator 'not' */
%right TNot

/* fixed the Decs problem with nonassoc */ 
%nonassoc TSemi
%nonassoc TId

/* Start symbol */
%start Prog

/******************************************************************************
 * Types returned by rules below 
 *****************************************************************************/

/* Program structure */
%type <AbSyn.Prog>        Prog
%type <AbSyn.FunDec list> FunDecs
%type <AbSyn.FunDec>      FunDec
%type <AbSyn.StmtBlock>   Block
%type <AbSyn.Dec list>    DBlock
%type <AbSyn.Stmt list>   SBlock
%type <AbSyn.Stmt list>   StmtSeq

/* Variables and parameter declarations, types */
%type <AbSyn.Dec list>    PDecl
%type <AbSyn.Dec list>    Params
%type <AbSyn.Dec>         Dec
%type <AbSyn.Dec list>    Decs
%type <AbSyn.Type>        Type

/* Statements */
%type <AbSyn.Stmt>        Stmt

/* Function and procedure parameters and index lists */
%type <AbSyn.Exp list>    CallParams
%type <AbSyn.Exp list>    Exps

/* L-Values and expressions */
%type <AbSyn.LVAL>        LVal
%type <AbSyn.Exp option>  Ret
%type <AbSyn.Exp>         Exp

%%

/******************************************************************************
 * Rules - a separate start rule is added automatically
 *****************************************************************************/

/* Program structure */
Prog    : TProgram TId TSemi FunDecs TEOF { $4 } /* $4 : AbSyn.FunDec list */

FunDecs : FunDecs FunDec { $1 @ [$2] }
        | FunDec         { [$1] }

FunDec  : TFunction TId TLParen PDecl TRParen TColon Type Block TSemi
            /* Func of Type * Ident * Dec list * StmtBlock * Pos */
            { AbSyn.Func ($7, #1 $2, $4, $8, getPos ()) }
        | TProcedure TId TLParen PDecl TRParen Block TSemi
            /* Proc of Ident * Dec list * StmtBlock * Pos */
            { AbSyn.Proc (#1 $2, $4, $6, getPos ()) }

Block   : DBlock SBlock { AbSyn.Block ($1, $2) }

DBlock  : TVar Decs { $2 } /* $2 : AbSyn.Dec list */
        |           { [] }

SBlock  : TBegin StmtSeq TSemi TEnd { $2 } /* $2 : AbSyn.Stmt list */
        | Stmt                      { [$1] } /* $1 : AbSyn.Stmt */

StmtSeq : StmtSeq TSemi Stmt { $1 @ [$3] }  /* AbSyn.Stmt list */
        | Stmt               { [$1] }


/* Variables and parameter declarations, types */
PDecl  : Params             { $1 } /* AbSyn.Dec list */
       |                    { [] }

Params : Params TSemi Dec   { $1 @ [$3] } /* AbSyn.Dec list */
       | Dec                { [$1] }

Dec    : TId TColon Type    /* AbSyn.Dec of Ident * Type * Pos */
                            { AbSyn.Dec (#1 $1, $3, getPos ()) }
                            
Decs   : Dec TSemi Decs     { $1 :: $3 }
       | Dec TSemi          { [$1] }

Type   : TInt               { AbSyn.Int (getPos ()) }
       | TChar              { AbSyn.Char (getPos ()) }
       | TBool              { AbSyn.Bool (getPos ()) }
       | TArray TOf Type    { AbSyn.Array ($3, getPos ()) }


/* Statements */
Stmt : TId TLParen CallParams TRParen  /* ProcCall of Ident * Exp list * Pos */
                                       { AbSyn.ProcCall (#1 $1, $3, getPos ()) }
     | TIf Exp TThen Block TElse Block
         /* IfThEl of Exp * StmtBlock * StmtBlock * Pos */
         { AbSyn.IfThEl ($2, $4, $6, getPos ()) }
     | TIf Exp TThen Block            
         /* IfThEl of Exp * StmtBlock * StmtBLock * Pos */
         { AbSyn.IfThEl ($2, $4, AbSyn.Block ([], []), getPos ()) }
     | TWhile Exp TDo Block            /* While of Exp * StmtBlock * Pos */
                                       { AbSyn.While ($2, $4, getPos ()) }
     | TReturn Ret                     /* Return of Exp option * Pos */
                                       { AbSyn.Return ($2, getPos ()) }
     | LVal TAssign Exp                /* Assign of LVAL * Exp * Pos */
                                       { AbSyn.Assign ($1, $3, getPos ()) }


/* Function and procedure parameters and index lists */
CallParams : Exps { $1 }
           |      { [] }

Exps       : Exp TComma Exps { $1 :: $3 }
           | Exp             { [$1] }


/* L-Values and expressions */
LVal : TId                          { AbSyn.Var (#1 $1) }
     | TId TLBracket Exps TRBracket { AbSyn.Index (#1 $1, $3) }

Ret  : Exp { SOME $1 } /* AbSyn.Exp option */
     |     { NONE }

Exp  : TNLit { AbSyn.Literal ( AbSyn.BVal (AbSyn.Num (#1 $1)), getPos ()) }
     | TBLit { AbSyn.Literal ( AbSyn.BVal (AbSyn.Log (#1 $1)), getPos ()) }
     | TCLit { AbSyn.Literal ( AbSyn.BVal (AbSyn.Chr (#1 $1)), getPos ()) }
     | TSLit { AbSyn.StrLit (#1 $1, getPos ()) }
     | TLCurly Exps TRCurly { AbSyn.ArrLit ($2, getPos ()) }
     | LVal { AbSyn.LValue ($1, getPos ()) }
     | TNot Exp { AbSyn.Not ($2, getPos ()) }
     | TLParen Exp TRParen { $2 }
     | TId TLParen CallParams TRParen { AbSyn.FunApp (#1 $1, $3, getPos ()) }

     | Exp TPlus  Exp { AbSyn.Plus ($1, $3, getPos ()) }
     | Exp TMinus Exp { AbSyn.Minus ($1, $3, getPos ()) }
     | Exp TTimes Exp { AbSyn.Times ($1, $3, getPos ()) }
     | Exp TSlash Exp { AbSyn.Div ($1, $3, getPos ()) }
     | Exp TEq    Exp { AbSyn.Equal ($1, $3, getPos ()) }
     | Exp TLess  Exp { AbSyn.Less ($1, $3, getPos ()) }
     | Exp TAnd   Exp { AbSyn.And ($1, $3, getPos ()) }
     | Exp TOr    Exp { AbSyn.Or ($1, $3, getPos ()) }

%%

(* mosmlyac trailer *)
\end{lstlisting}

\section{Grammer Tester script}
\begin{lstlisting}[style=paladim, caption=GrammaTester script]
load "Obj";
load "Parsing";
load "Lexing";
load "Char";

load "AbSyn";
load "Parser";
load "Lexer";

fun readFile s = let
  val is = TextIO.openIn s
  val str = TextIO.inputAll is  
in
  str 
end


fun parser s = let val buf = Lexing.createLexerString s
                  in Parser.Prog Lexer.Token buf
end

fun parse s = let
  val str = readFile s
in
    parser str
end
\end{lstlisting}


\end{document}
