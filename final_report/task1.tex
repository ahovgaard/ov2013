\section{Parser implementation in \texttt{mosmlyac}}
A large part of this milestone assignment is the implementation of a parser for
the toy language \textsf{Paladim}, using the parser generator \texttt{mosmlyac}.
The code for the \texttt{mosmlyac} parser generator is included in appendix
\ref{app:grm}.\\
\\
The file ``Parser.grm'' declares a number of token types that represent the
different tokens, i.e. entities or objects, of the \textsf{Paladim} language.
The lexical analyser of the compiler, as generated using \texttt{mosmllex} and
specified in ``Lexer.lex'', returns a list of tokens of this type and this is
given as input to the parser. To do this, the lexer code was also changed to
produce the types defined in the new \texttt{mosmlyac} parser, rather than the
ones defined in the LL(1) parser, which basically just involves replacing all
occurences of \emph{LL1Parser} with \emph{Parser}, in the file ``Lexer.lex''.\\
The grammar rules, given in figure 3 of the group project assignment
specification (``GroupProject.pdf''), is rewritten in a \texttt{mosmlyac}
specific way of specifying the grammar of the language, but besides from minor
syntactical differences, they are largely identical.\\
\\
Our main goal was to create the same language as the LL(1) parser, and then
connect the new parser to the rest the compiler(Changing the Driver.sml to use
our parser instead). 
We started with the grammar, which was given with the assignment. We then
modified the grammar to conform with \texttt{mosmlyacc}, and also modified
certain areas of the grammar that was giving conflicts (you can see the
conflicts we had in the 'Resolving parse conflicts' section).\\
\\
We also changed some of the given grammar that we thought could be better
expressed with some new grammar, but still did that same thing as the given
grammar (see the 'Changes to the \textsf{Paladim} grammar' section for that).\\
\\
When we were done with the grammar we also had to make sure we used the correct
return types, we used the ``AbSyn.sml'' file to check for the the datatype types
to use, and we also looked in the LL(1) parser to see what types that parser
used for better understanding of the types.


\subsection{Resolving parse conflicts}
We fixed the dangling-else ambiguity by making the \texttt{then} and
\texttt{else} symbols right-associative and giving them the same precedence,
i.e. by using the following mosmlyac declaration:

\begin{verbatim}
  %right TThen TElse 
\end{verbatim}

\noindent
We had a shift/reduce conflict with the modified productions of the $Decs$
nonterminal (which you can see in the changes to the paladin grammar section).
The way we solved that problem was to force a reduce over a shift by making
\texttt{TId} and \texttt{TSemi} non-associative, which fixed the conflict
problem. \\
\\
We also received countless reduce conflicts from the rewritten operation
expressions (see 'Changes to \textsf{Paladim} grammar' section).  This was
because the parser did not know which reduction it should choose, e.g. 2+3*3,
Plus(2, Times(3,3)) or Times(Plus(2, 3), 3). To fix this we introduced
precedence and associativity for each operation and that fixed the problem.
Please see the appendix for the precedence and associativity assignments for
each operation in the ``Parser.grm'' code.

\subsection{Changes to the \textsf{Paladim} grammar}
We removed the OP non terminal and instead created all the cases for the
different operations instead.
\begin{lstlisting}[style=GrammarStyle]
  $Exp\;$  ->  $Exp$ $OP$ $Exp$ 
  $OP$   ->  TPlus | TMinus | TTimes | TSlash 
  $OP$   ->  TEq | TLess | TAnd | TOr
\end{lstlisting}
to
\begin{lstlisting}[style=GrammarStyle]
  $Exp$  ->   $Exp$ TPlus  $Exp$
  $Exp$  ->   $Exp$ TMinus $Exp$
  $Exp$  ->   $Exp$ TTimes $Exp$
  $Exp$  ->   $Exp$ TSlash $Exp$
  $Exp$  ->   $Exp$ TEq    $Exp$
  $Exp$  ->   $Exp$ TLess  $Exp$
  $Exp$  ->   $Exp$ TAnd   $Exp$
  $Exp$  ->   $Exp$ TOr    $Exp$ 
\end{lstlisting}
\hfill\\
We decided to remove the left recursion because of a reduce/shift conflict.
\begin{lstlisting}[style=GrammarStyle]
  $Decs$  ->    $Decs$ $Dec$ ; 
  $Decs$  ->    $Dec$ ;
\end{lstlisting}
to
\begin{lstlisting}[style=GrammarStyle]
  $Decs$   ->   $Dec$ ; $Decs$
  $Decs$   ->   $Dec$ ;
\end{lstlisting}


\subsection{Testing methodology}
At the current point in the space-time continuum we are only able to compile a
subset of the programs included in the \textbf{DATA}-folder. Those programs
are:\\
\begin{enumerate}
  \item fibRex.pal
  \item fibWhile.pal
  \item proctest.pal
  \item readtest.pal
  \item shortest.pal
\end{enumerate}

Also whenever we changed something new in the grammar, e.g. modified some
precedence for the Expression, we would use our grammar script which allows us
to create the AbSyn syntax tree without it being further processed. This was
quite handy since we can actually test things that aren't implemented yet for
the compiler, typecheck, etc. A good example of the usage of the grammar script
was when we had to test the 'or' and 'not' operators. Since the compiler still
does not support these operations, we had to use this script to see if our
precedence was correct - which it is.
\\
\\
Whenever we had made some major changes to the parser, we would run all the
aforementioned pal files in interpreter- and compile-mode to check if everything
still worked. Before we even started coding the grammar for the parser, we saw
how the LL(1) parser handled the aforementioned pal files, and since we are
building a replacement for the LL(1) parser, our new parser should at least
handle the same files and give the same output as the LL parser. It was
therefore fairly easy to see if the parser tests were correct, regarding the pal
files. Pal files like the fibWhile and fibRec were also quite easy to test,
since they took some input and gave some output, which made it alot more easy to
make test cases.
