\section{Task 4: Type Checking and Code Generation for Array Indexing}

To allow the ability of array, we had to implement the ability to index, both in the type check and the compiler. We first decided to implement
the type checking, since we couldn't test our indexing without the type checking.

\begin{lstlisting}[style=MLStyle, caption=type checking for indexing]		
| typeCheckExp( vtab, AbSyn.LValue( AbSyn.Index(id, inds), pos ), _ ) =
    let 
      val new_inds = map (fn ind => typeCheckExp(vtab, ind, KnownType(BType Int))) inds
      val ind_tps = map (fn ind => typeOfExp ind) new_inds
      val indAmount = List.length ind_tps
      val areInts = foldl (fn (x, y) => typesEqual(BType Int, x) andalso y) true ind_tps
      val id_tp = SymTab.lookup id vtab
    in
      case id_tp of 
        NONE => raise Error("The array does not exist in the vtab.", pos)
      | SOME tp => 
        let
          val rank = 
					case tp of
								Array(r, _) => r
                | _ => raise Error("typechecking Index, variable must be an array", pos)
        in
          if rank > 0 andalso rank = indAmount andalso areInts 
          then LValue(Index((id, tp), new_inds), pos)
          else raise Error("array " ^ id ^ " does not have the right parameters", pos)
        end
    end	
\end{lstlisting}
The implementation compiles every index expression expecting that every index is an integer, which is the only way to represent indexes in 
paladim. We use typesEqual to make sure that every index is integer and we have to look up the variable name in vtable to get the base address of the index. 
We also need to use the rank to make sure that it's under zero and that it equals the length of the indices. If these condition are not meet we raise an error. 
See listing 1 for the implementation.
\\
\\
In the implementation of the indexing for compiler, we made a decision to try and exploit the fact the arrays are static, i.e. we can't change
the rank of the array on runtime. This meant we could unroll the loop to check the dimensions, which we otherwise would have to make a loop in
Mips if our arrays were dynamic. We made a decision to split up every part of the Mips code.
First we got all the list dim mip code, we then got list stride code, we than used the dims list to check the bounds, 
after this we used the stride list to get the index, then added that index together with the base address of the array. 
This meant we had to use a lot of variable registers. In retrospect we feel that making another version that only used a few variable 
registers would most likely have been a better solution, since it's more likely to get spills if the compiler uses too many variable registers(if the register allocator implements a spilling algorithm, 
which it would have to, otherwise the assembler code wouldn't work at all). It was also important to make sure what type the array had, since the size of a character is 1 byte and the size of an int is 4 byte, 
we therefore had to check the size of the type to generate the correct Mips.MULT. An important part in calculating the index was remembering that the stride of the last one was 1 and was not a part of the
list of strides. Which meant that the length of our expression list was one expression larger than the strides list, which meant we did a pattern match for when that was the case and just added the expression value with
the current accumulated value of the index.
See listing 2 for the implementation of indexing for the compiler.
\begin{lstlisting}[style=MLStyle, caption=Implementation of indexing for the compiler]
	| compileLVal( vtab : VTab, Index ((n, Array(rank, tp)),inds) : LVAL, pos : Pos ) =
      let 
        val mem = case SymTab.lookup n vtab of 
                    SOME n => n
                  | NONE => raise Error("no such variable for array, " ^ n, pos)
        val strideEnd = (rank * 2)-1
        val ptr = rank * 2 *4
        fun dims base i  = 
          let
            val r = "metaDataDims_"^newName()
            val code = Mips.LW(r, base, Int.toString (i*4))
          in
            if i < rank then
              [(code, r)] @ (dims base (i+1))
            else
              []
          end
        fun stride base i =
         let
         val r = "metaDataStride_"^newName() 
         val code = Mips.LW(r, base, Int.toString (i*4))
         in
          if i < strideEnd then
              [(code, r)] @ (stride base (i+1))
            else
              []
          end

        fun calculateIndices [] = []
          | calculateIndices (exp::exps) =
         let 
           val temp = "indices_"^newName()
           val code = compileExp(vtab, exp, temp)
         in
           [(code, temp)] @ calculateIndices exps
         end

        val arrayAddress = "metaPointer_"^newName()
        val addressCode = [Mips.ADDI(arrayAddress, mem, makeConst ptr)]
        val strideMeta = stride mem rank          
        val dimsMeta = dims mem 0 
        val compiledInds = calculateIndices inds

        fun checkBounds _ [] = []
          | checkBounds [] _ = []
          | checkBounds ((c, r)::dmeta) ((_, r2)::exps) =
          let
            val tb = "bounds_"^newName()
            val code = [Mips.ADDI (tb, r2, "1"), 
                        Mips.SUB(tb, r, tb), 
                        Mips.SLTI(tb, tb, "0"), 
                        Mips.BNE(tb, "0", "_IllegalArrIndexError_") ]
            in
              [c] @ code @ (checkBounds dmeta exps)
            end

        fun computeIndexAddress [] ((c,r)::[]) place = c @ [Mips.ADD(place, r, place)] 
          | computeIndexAddress ((c, r)::smeta) ((_, r2)::exps) place =
          let
            val tTemp = "index_"^newName()
            val code = [c, Mips.MUL(tTemp, r2, r), Mips.ADD(place, tTemp, place)] 
          in
            code @ computeIndexAddress smeta exps place
          end
          | computeIndexAddress _ _ _ = raise Error("this cannot happen", pos)

        val multValue = (case tp of
                          Int => 4
                        | Char => 1
                        | Bool => 1)
                          
        val tMult = "multIndex_"^newName()
        val tResult = "index_"^newName()
        val init = [Mips.LI(tResult, "0")]
        val checkCode = checkBounds dimsMeta compiledInds 
        val indexCode = computeIndexAddress strideMeta compiledInds tResult 
        val computeFinalAddress = [Mips.ADDI(tMult, "0", makeConst multValue),
        Mips.MUL(tResult, tResult, tMult), Mips.ADD(arrayAddress, arrayAddress, tResult)] 
  
        val indCode = foldl op @ [] (map (fn (x,y) => x) compiledInds)
        val result = addressCode @ indCode @ checkCode @ 
        init @ indexCode @ computeFinalAddress 
      in
        (result, Mem (arrayAddress)) 
      end	
\end{lstlisting}



\subsection{Testing}
We tested our code using the test test3D.pal and testIndex.pal which uses a bunch of arrays, which tested the indexing quite well, 
so well in fact that we found a bug in our compiler code, where we forgot to set the index register to zero before using it, 
causing memory out of bounds. Since we also had out of bounds checks, we also had to test our indexing against invalid arguments. 
We also had to test the Type, but type checking in index was also a requirement for the indexing to work, 
which meant that we tested them together with the indexing. The only thing we didn't test together was the error checking in type, 
so we also tested wrong ranks, inds being longer than rank, and inds not being integer. The aforementioned properties to test can be
seen in the listings 3-6.
\\
\\
Listing 3 is a paladim program that allowed us to write in the indexes we wanted to check, since bound checking is done on runtime this was
a quite good way of checking if our indexing worked, and also allowed us to check bounds.

\begin{lstlisting}[style=paladim, caption=testIndexBounds.pal tests bounds]
program bounds;
procedure main()
    var x : array of array of array of int;
        a : int;
        b : int;
        c : int;
        result : int;
begin
    x := {{{1, 2}, {3, 4}},{{5, 6}, {7, 8}}};
    write("which index do you want?");
    write('\n');
    write("array is {{{1, 2}, {3, 4}},{{5, 6}, {7, 8}}}");
    write('\n');

    a := read();
    b := read();
    c := read();
    result := x[a,b,c];
    write(result);
end;
\end{lstlisting}
Listing 4-6 are paladim programs that are meant to fail at compile time since they are all type checking errors.

\begin{lstlisting}[style=paladim, caption=testWrongIndexType.pal tests invalid types in indexing]
program bounds;

procedure main()
    var x : array of array of array of int;
begin
    x := {{{1, 2}, {3, 4}},{{5, 6}, {7, 8}}};
    x[1,'a',true] := 2;
end;
\end{lstlisting}

\begin{lstlisting}[style=paladim, caption=testZeroRank.pal tests zero rank]
program bounds;

procedure main()
    var x : int;
begin
    x[0,0,0] := 2;
end;
\end{lstlisting}

\begin{lstlisting}[style=paladim, caption=testWrongRank.pal tests wrong rank]
program bounds;
procedure main()
    var x : array of array of array of int;
begin
    x := {{{1, 2}, {3, 4}},{{5, 6}, {7, 8}}};
    x[1,2,3,4] := 2;
end;
\end{lstlisting}
Using the test that we just mentioned, we can be quite sure that the implementation works, since all the tests passed.

