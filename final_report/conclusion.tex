\section{Conclusion}
%Using the knowledge of the LL(1) parser (given from the assignment) and that of
%yacc, we rewrote the grammar from the given grammar. Using our knowledge of LR
%parsers to handle shift/reduce conflicts, we used associativity to handle
%reduce/shift conflicts. By testing all of our changes to the parser, we can be
%quite certain that it actually works as intended, and even though we didn't do
%really structured test cases, we did test quite a lot. We think in the later
%parts of the project a more structured testing approach would be better, but we
%felt that the tests were quite simple for the parser, and therefore testing of
%the already given pal files and our grammarTester script were enough.
We were given a unfinished compiler for a toy language, called Paladim, and our task was to
develop it further, adding certain features, such as support for arrays,
call-by-value-result, type-inference, integer multiplication and division,
boolean operators and otherwise expanding the given codebase. As previously said,
the codebase was already started on, but some changes and additions were
necessary to do.
These changes could range from just changing a line of code, to adding up
several lines of new code.\\
Our testing methodology is mainly if not only, manual. We used the supplied
paladim files in the DATA folder and added a few more files on our own, if those
tests passed, then great, otherwise back to the drawing board.\\
Overall the process has been forfilling and great learning experience, showing
many different aspects of creating a language, such as parsing, interpretation
and machine-code generation.\\

