\section{Task 2: Integer Multiplication and Division, and Boolean Operators}
The second task was to implement integer multiplication and division, and the
boolean operators \verb|or| and \verb|not|.\\
\\
The constructors for expressions of these types were added (uncommented) to the
typed abstract syntax tree modules, \verb|TpAbsyn.sml|, that is, by adding the
following lines of code (see figure \ref{fig_expConstructors}) to the \verb|Exp|
datatype declaration.
\begin{figure}[H]
  \begin{lstlisting}[style=MLStyle]
  | Times of Exp * Exp * Pos  (* e.g., x * 3 *)
  | Div   of Exp * Exp * Pos  (* e.g., x / 3 *)         
  | Or    of Exp * Exp * Pos  (* e.g., (x=5) or y *)
  | Not   of Exp       * Pos  (* e.g., not (x>3) *)      
  \end{lstlisting}
  \caption{Expression constructors.}
  \label{fig_expConstructors}
\end{figure}

\noindent
Cases for these constructors were added to the pretty printing function for
expressions \verb|pp_exp| in the same file, with the code shown in figure
\ref{fig_expPP}.
\begin{figure}[H]
  \begin{lstlisting}[style=MLStyle]
  | pp_exp (Or    (e1, e2, _)) = "( " ^ pp_exp e1 ^ " ^ " ^ pp_exp e2 ^ " )"
  | pp_exp (Times (e1, e2, _)) = "( " ^ pp_exp e1 ^ " * " ^ pp_exp e2 ^ " )"
  | pp_exp (Div   (e1, e2, _)) = "( " ^ pp_exp e1 ^ " / " ^ pp_exp e2 ^ " )"
  | pp_exp (Not   (e1, _))     = "not ( " ^ pp_exp e1 ^" )"
  \end{lstlisting}
  \caption{Cases for pretty printing Exp constructors.}
  \label{fig_expPP}
\end{figure}

\noindent
Similarly cases were added to the \verb|typeOfExp| function, which returns the
type of the expression, and \verb|posOfExp|, which just returns the position
given in the last component of the tuple value of the constructors. These
changes are shown in figure \ref{fig_expTyp} and \ref{fig_expPos}.
\begin{figure}[H]
  \begin{lstlisting}[style=MLStyle]
  | typeOfExp (Times (a, b, _)) = typeOfExp a
  | typeOfExp (Div   (a, b, _)) = typeOfExp a
  | typeOfExp (Or    (_, _, _)) = BType Bool
  | typeOfExp (Not   (_, _)   ) = BType Bool
  \end{lstlisting}
  \caption{Cases for type of expression constructors.}
  \label{fig_expTyp}
\end{figure}

\begin{figure}[H]
  \begin{lstlisting}[style=MLStyle]
  | posOfExp (Times (_, _, p)) = p
  | posOfExp (Div   (_, _, p)) = p
  | posOfExp (Or    (_, _, p)) = p
  | posOfExp (Not   (_,    p)) = p
  \end{lstlisting}
  \caption{Cases for positions of expression constructors.}
  \label{fig_expPos}
\end{figure}

\noindent
To make these expressions available for interpretation, two functions
\verb|evalOr| and \verb|evalNot| are created and cases for all four new
expressions are added to the evalExp function. These cases recursively calls
\verb|evalExp|, evaluating the expressions given as arguments to the
multiplication, division or boolean operators, and passes along the symbol
tables. The changes are made in the interpreter module, \verb|TpInterpret.sml|,
see figure \ref{fig_expInterpret}.
\begin{figure}[H]
  \begin{lstlisting}[style=MLStyle]
  fun evalOr (BVal (Log b1), BVal (Log b2), pos) = BVal (Log(b1 orelse b2))
    | evalOr (v1, v2, pos) =
        raise Error( "Or: argument types do not match. Arg1: " ^
                      pp_val v1 ^ ", arg2: " ^ pp_val v2, pos )

  fun evalNot (BVal (Log b), pos) = BVal (Log(not b))
    | evalNot (v1, pos) =
        raise Error( "Not: argument type is wrong. Arg: " ^
                      pp_val v1, pos)

  (* ... *)

  (* fun evalExp ... *)
    | evalExp ( Times(e1, e2, pos), vtab, ftab ) =
        let val res1 = evalExp(e1, vtab, ftab)
            val res2 = evalExp(e2, vtab, ftab)
        in  evalBinop(op *, res1, res2, pos)
        end
    | evalExp ( Div(e1, e2, pos), vtab, ftab ) =
        let val res1 = evalExp(e1, vtab, ftab)
            val res2 = evalExp(e2, vtab, ftab)
        in  evalBinop(op div, res1, res2, pos)
        end

  (* ... *)

    | evalExp ( Or(e1, e2, pos), vtab, ftab ) =
          let val r1 = evalExp(e1, vtab, ftab)
              val r2 = evalExp(e2, vtab, ftab)
    in  evalOr(r1, r2, pos)
    end

    | evalExp ( Not(e1, pos), vtab, ftab ) =
          let val r1 = evalExp(e1, vtab, ftab)
    in evalNot(r1, pos)
    end
  \end{lstlisting}
  \caption{Evaluation of expressions in interpreter.}
  \label{fig_expInterpret}
\end{figure}

\noindent
The final changes in this task were made in the \verb|Type.sml| file: four
cases were added to the function \verb|typeCheckExp|, which checks that the
types of multiplication, division or boolean operator expressions are correct.
This is done be recursively deciding the types of subexpressions and checking
that these match with the expected types. The changes are shown in figure
\ref{fig_expTypeCheck}.
\begin{figure}[H]
  \begin{lstlisting}[style=MLStyle]
  | typeCheckExp (vtab, AbSyn.Times (e1, e2, pos), _) =
      let val e1_new = typeCheckExp(vtab, e1, KnownType(BType Int))
          val e2_new = typeCheckExp(vtab, e2, KnownType(BType Int))
          val (tp1, tp2) = (typeOfExp e1_new, typeOfExp e2_new)
      in  if typesEqual(BType Int, tp1) andalso typesEqual(BType Int, tp2)
          then Times(e1_new, e2_new, pos)
          else raise Error("in type check times exp, one argument is not \
                           \of int type " ^ pp_type tp1 ^ " and " ^
                           pp_type tp2 ^ " at ", pos)
      end

  | typeCheckExp (vtab, AbSyn.Div (e1, e2, pos), _) =
      let val e1_new = typeCheckExp(vtab, e1, KnownType(BType Int))
          val e2_new = typeCheckExp(vtab, e2, KnownType(BType Int))
          val (tp1, tp2) = (typeOfExp e1_new, typeOfExp e2_new)
      in  if typesEqual(BType Int, tp1) andalso typesEqual(BType Int, tp2)
          then Div(e1_new, e2_new, pos)
          else raise Error("in type check div exp, one argument is not \
                           \of int type " ^ pp_type tp1 ^ " and " ^
                           pp_type tp2 ^ " at ", pos)
      end

  | typeCheckExp (vtab, AbSyn.Or  (e1, e2, pos), _) =
      let val e1_new = typeCheckExp(vtab, e1, KnownType(BType Bool))
          val e2_new = typeCheckExp(vtab, e2, KnownType(BType Bool))
          val (tp1, tp2) = (typeOfExp e1_new, typeOfExp e2_new)
      in  if typesEqual(BType Bool, tp1) andalso typesEqual(BType Bool, tp2)
          then Or(e1_new, e2_new, pos)        
          else raise Error("in type check and exp, one argument is not \
                           \of bool type " ^ pp_type tp1 ^ " and " ^
                           pp_type tp2 ^ " at ", pos)
      end

  | typeCheckExp (vtab, AbSyn.Not (e1, pos), _) =
      let val e1_new = typeCheckExp(vtab, e1, KnownType(BType Bool))
          val tp = typeOfExp e1_new
      in  if typesEqual(BType Bool, tp)
          then Not(e1_new, pos)
          else raise Error ("in type check and exp, the only argument is \
                            \not of bool type " ^ pp_type tp ^ " at" , pos )
      end
  \end{lstlisting}
  \caption{Type checking of expressions.}
  \label{fig_expTypeCheck}
\end{figure}


\subsection{Testing}
Apart from the included files in the DATA folder, we have made a few paladim
files that exclusively target the implemented functions, those are as follows:
\\
\begin{itemize}
  \item Integer multiplication (*)
  \item Integer division (/)
  \item Boolean operator (=, not, or, and)
\end{itemize}

We initialize variables, assign values to them and then test them.
\begin{lstlisting}[style=paladim]
program test_task2;

function intMul(a : int; b : int) : int
begin
  return a*b;
end;

function intDiv(c : int; d : int) : int
var e : int;
begin
  e := c/d;
  return e;
end;

procedure main()
var x : int;
    z : int;
    n : int;
    m : int;
    a : bool;
    b : bool;
    c : bool;
begin
  x := 10;
  z := 9;
  n := intMul(5, 2);
  write("\n");
  write("5 * 2 = ");
  write(n);
  m := intDiv(10, 6);
  write("\n");
  write("10 / 6 = ");
  write(m);
  write("\n");
  write("not true = ");
  b := not true;
  write(b);
  write("\n");
  a := not false;
  c := a or not a;
  write("a or not a = ");
  write(c);
  write("\n");
  c := a and not a;
  write("a and not a = ");
  write(c);
  write("\n");
  b := c or not a = b and a = c;
end;
\end{lstlisting}
