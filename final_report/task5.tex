\section{Task 5}
The fifth task was to implement call-by-value-result semantics for procedures,
which means that procedures work on a local copy of its arguments, but copy the
values of these back into the given arguments when it returns.

\subsection{Interpreter}
This parameter passing strategy is implemented in both the \textsf{Paladim}
interpreter and in the \textsc{Mips} code generator. In the interpreter module,
\verb|TpInterpret.sml|, the \verb|callFun| function was updated in the case when
the function identifier is predefined and the  option return type is
\verb|NONE|, i.e. when the called function is a procedure without a return a
value.\\
In this case, a new function \verb|updateOuterVtable| is applied to each pair of
actual expression argument and formal argument of the function call, using the
code shown in figure \ref{fig_callFun}
 
\begin{figure}[H]
  \begin{lstlisting}[style=MLStyle]
and callFun ( (rtp : Type option, fid : string, fargs : Dec list,
               body : StmtBlock, pdcl : Pos),
              aargs : Value list, aexps : Exp list, vtab, ftab, pcall : Pos
            ) : Value option =
  case fid of
       (* treating "special" functions such as ord/chr/write, etc. *)

  (* ... *)

     | other    =>
         let val new_vtab = bindTypeIds(fargs, aargs, fid, pdcl, pcall)
             val res  = execBlock( body, new_vtab, ftab )
         in case (rtp, res) of
                 (NONE  , _)      => (app (updateOuterVtable vtab new_vtab)
                                          (ListPair.zip (aexps, fargs));
                                     NONE)
               | (SOME t, SOME r) => (* ... *)
         end
  \end{lstlisting}
  \caption{Case for procedure call in \texttt{callFun}.}
  \label{fig_callFun}
\end{figure}

\noindent
The function \verb|updateOuterVtable| looks up the identifiers of the actual
argument and the formal argument, and if these both exist, the formal argument
reference is assigned the dereferenced value of the actual argument, using the
imperative SML references feature. The \verb|updateOuterVtable| function is
shown in figure \ref{fig_updateOuterVtable}.
 
\begin{figure}[H]
  \begin{lstlisting}[style=MLStyle]
(* Update the outer vtable with data from the inner vtable. Here, out_exp is
 * an expression in a call, e.g. x in f(x), and in_arg is the local argument
 * name in f, e.g. m when 'procedure f(m) ... end' is defined. Remember that
 * call by value result requires that argument expressions are variable names,
 * i.e. expressions like '2 + x' do not work, since '2 * x' is not an LValue
 * variable name.
 *)
and updateOuterVtable vtabOuter vtabInner (LValue out_exp, Dec((s, t), pos)) = 
  let
    fun getIdent (Var((s, t)), _)      = s
      | getIdent (Index((s, t), _), _) = s
    
    fun getPos (Var((s, t)), pos)      = pos
      | getPos (Index((s, t), _), pos) = pos

    val ident = getIdent out_exp
    val finalValue  = SymTab.lookup s vtabInner
    val modifyValue = SymTab.lookup ident vtabOuter
  in
    case (modifyValue, finalValue) of
         (NONE, NONE)      =>
            raise Error("Could not find the final value for call by value \
                        \result, " ^ s, getPos(out_exp))
       | (SOME m, SOME f)  => (m := !f; ())
       | (_, _)            =>
            raise Error("Could not find the final value for call by value \
                        \result, " ^ s, getPos(out_exp))
  end

  | updateOuterVtable _ _ _ = ()
  \end{lstlisting}
  \caption{Function \texttt{updateOuterVtable}.}
  \label{fig_updateOuterVtable}
\end{figure}

\subsection{\textsc{Mips} code generation}
To make the call-by-value-result strategy work in code generation, a few changes
were made a few places in the \verb|Compiler.sml| file. The \verb|putArgs|
function, recursively compiling expressions given as arguments to a procedure
and generating \textsc{Mips} \verb|MOVE| instruction for moving arguments into
registers for the procedure to work on, was modified to also generate and return
a procedure stack frame prologue.



\subsection{Testing}
