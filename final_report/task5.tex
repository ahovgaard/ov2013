\section{Task 5}
The fifth task was to implement call-by-value-result semantics for procedures,
which means that procedures work on a local copy of its arguments, but copy the
values of these back into the given arguments when it returns.\\
\\
This parameter passing strategy is implemented in both the \textsf{Paladim}
interpreter and in the \textsc{Mips} code generator. In the interpreter module,
\verb|TpInterpret.sml|, the \verb|callFun| function was updated in the case when
the function identifier is predefined and the  option return type is
\verb|NONE|, i.e. when the called function is a procedure without a return a
value.\\
In this case, a new function \verb|updateOuterVtable| is applied to each pair of
actual expression argument and formal argument of the function call, using the
code shown in figure \ref{fig_callFun}
 
\begin{figure}[H]
  \begin{lstlisting}[style=MLStyle]
and callFun ( (rtp : Type option, fid : string, fargs : Dec list,
               body : StmtBlock, pdcl : Pos),
              aargs : Value list, aexps : Exp list, vtab, ftab, pcall : Pos
            ) : Value option =
  let val () = () (* print ("\nCalling function: "^fid^"\n") *)
  in  case fid of
        (* treating "special" functions such as ord/chr/write, etc. *)

(* ... *)

      | other    =>
          let val new_vtab = bindTypeIds(fargs, aargs, fid, pdcl, pcall)
              val res  = execBlock( body, new_vtab, ftab )
          in  ( case (rtp, res) of
                  (NONE  , _) => (app (updateOuterVtable vtab new_vtab)
                                      (ListPair.zip (aexps, fargs));
                                 NONE)
                | (SOME t, SOME r) => if   typesEqual(t, typeOfVal r) 
                                      then SOME r
                                      else raise Error("in call fun: result does " ^
                                                       "not match the return type! In fun/proc:" ^ fid ^ 
                                                       " rettype: "^pp_type t^" result: "^pp_val r, pcall )
                | otherwise        => raise Error("in call: fun/proc "^fid^" illegal result: ", pcall ) )
          end
    end
  \end{lstlisting}
  \caption{Case for procedure call in callFun.}
  \label{fig_callFun}
\end{figure}
 




The lists of actual expression arguments and formal arguments are w

A function \verb|updateAll| walks through the lists actual expressiong
arguments and formal arguments, and calls a new function
\verb|updateOuterVtable| on each pair.

\subsection{Testing}
